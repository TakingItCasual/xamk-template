\section{Getting Started}

\subsection{"What am I supposed to do with these files?"}

The .tex files are what you edit.
What you want to get out of them is a PDF file.
This requires a \LaTeX\ compiler.
If you just want to get started and avoid The Complicated Stuff I'd recommend a free online compiler such as \href{https://www.overleaf.com/}{Overleaf}.
Simply take the provided zip file and upload it with \textbf{New Project | Upload Project}.
Under the project menu, configure the TeX Live version to the latest.

\subsubsection{The Complicated Stuff}

For editing and compiling \LaTeX\ projects locally on your computer, I'd recommend VS Code with the \href{https://marketplace.visualstudio.com/items?itemName=James-Yu.latex-workshop}{LaTeX Workshop} extension \parencite{workshop}.
You'll also need a \TeX\ distribution (I'd recommend \href{https://www.tug.org/texlive/acquire-netinstall.html}{TeX Live} \parencite{texlive}).
If you don't do a full install, you'll need to manually install packages as compilation errors crop up.

For the minted package (used for code syntax highlighting), Python needs to be installed along with Python's \textbf{Pygments} package \parencite{pygmentize}.

Specify the following VS Code settings to enable the usage of the minted package, and to place generated files into an \texttt{out/} subfolder (try deleting this subfolder if you are experiencing weird compilation issues).
\begin{code}{json}[label=code:vscode-config]
"latex-workshop.latex.outDir": "%DIR%/out",
"latex-workshop.latex.tools": [{
    "name": "latexmk",
    "command": "latexmk",
    "args": ["--shell-escape", "-synctex=1", "-interaction=nonstopmode", "-pdf", "-outdir=%OUTDIR%", "%DOC%"]
}],
"latex-workshop.latex.recipes": [{
    "name": "latexmk",
    "tools": ["latexmk"]
}]
\end{code}

Restart VS Code, then with VS Code open the \texttt{xamk-template} folder.
Open the \texttt{xamk-template.tex} file.
A \TeX\ view should appear in the side bar.
Select \textbf{\TeX\ | Build LaTeX Project | Recipe: latexmk}, wait for the compilation to finish, then view the PDF with \textbf{\TeX\ | View LaTeX PDF}.

\subsection{Formatting}

{\centering\sc \large A Simple Sample \LaTeX\ File \parencite{formatting}\par}
\centerline{\sc Stupid Stuff I Wish Someone Had Told Me Four Years Ago}
\centerline{\it (Read the chapter1.tex file along with this or it won't make much sense)}

The first thing to realize about \LaTeX\ is that it is not ``WYSIWYG''.
In other words, it isn't a word processor; what you type into your .tex file is not what you'll see in your PDF.
For example, \LaTeX\ will      completely     ignore               extra    spaces    within                             a line of your .tex file.
Pressing return
in
the
middle
of
a
line
will not register in your PDF.
However, a double carriage-return is read as a paragraph break.

Like this.
But any carriage-returns after the first two will be completely ignored; in other words, you


can't

add






more




space


between




lines, no matter how many times you press return in your .tex file.

In order to add vertical space you have to use ``vspace''; for example, to get three lines of space you would type \verb|\vspace{3pc}| (``pc'' stands for ``pica'', a font-relative size), like this:
\vspace{3pc}

Notice that \LaTeX\ commands are always preceded by a backslash.
Some commands, like \verb|\vspace|, take arguments (here, a length) in curly brackets.

The second important thing to notice about \LaTeX\ is that you type in various ``environments''.
So far we've just been typing regular text (except for a few inescapable usages of \verb|\verb| and the centered, smallcaps, large title).
There are basically two ways that you can enter and/or exit an environment;
\vspace{0pc}

\centerline{this is the first way...}

\begin{center}
	this is the second way.
\end{center}

\noindent Actually there is one more way, used above; for example, {\sc this way}.
The way that you get in and out of environment varies depending on which kind of environment you want; for example, you use \verb|\underline| ``outside'', but \verb|\it| ``inside''; notice \underline{this} versus {\it this}.

The real power of \LaTeX\ (for us) is in the math environment.
You push and pop out of the math environment by typing \verb|$|.
For example, $2x^3 - 1 = 5$ is typed between dollar signs as \verb|$2x^3 - 1 = 5$|.
Perhaps a more interesting example is $\lim_{N \to \infty} \sum_{k=1}^N f(t_k) \Delta t$.

You can get a fancier, display-style math environment by enclosing your equation with double dollar signs.
This will center your equation, and display sub- and super-scripts in a more readable fashion:

$$\lim_{N \to \infty} \sum_{k=1}^N f(t_k) \Delta t$$

If you don't want your equation to be centered, but you want the nice indicies and all that, you can use \verb|\displaystyle| and get your formula ``in-line''; using our example this is $\displaystyle \lim_{N \to \infty} \sum_{k=1}^N f(t_k) \Delta t$.
However, this can screw up your line spacing a little bit.

There are many more things to know about \LaTeX\ and we can't possibly talk about them all here.
You can use \LaTeX\ to get tables, commutative diagrams, figures, aligned equations, cross-references, labels, matrices, and all manner of strange things into your documents.
You can control margins, spacing, alignment, {\it et cetera} to higher degrees of accuracy than the human eye can perceive.
You can waste entire days typesetting documents to be ``just so''.
\LaTeX\ can do anything, given you can find the package for the task.

The best way to learn \LaTeX\ is by example.
Get yourself a bunch of .tex files, see what kind of output they produce, and tinker with them to learn how they work.
The internet is full of instructions and examples for how to do various things in \LaTeX\ (one good resource is \url{https://tex.stackexchange.com}).
You'll need to do a lot of copy-pasting while learning how to use \LaTeX.
Good luck!

% Notice how each sentence starts from a new line, this helps with version control.
% Also notice the occasional ``<text>'', which creates the fancier quotation marks.