This is a model for the English thesis abstract.
It must be short enough as to not push the abstract table into the next page.
The abstract is typically made up of 3 to 5 paragraphs.

The abstract table begins with identification data in the correct fields, as presented in this template: author’s name, official name of Degree, month and year of approval, supervisors’ names with the official job titles (supervising teacher, the company name or the company’s representative, if the study was commissioned), number of pages from the title page to the end of the list of references / list of figures or tables, and the pages of appendices.
The first word in the title has capital initial.
Otherwise, the title has lower-case initial letters, except for words that are capitalized for grammatical reasons.

The abstract is a concise, independent account of the thesis content.
It has about 200-250 words that summarize the main content.
It introduces the purpose, methods and key results and conclusions.
Background information is not included, unless it is especially relevant for fully understanding the text without reading any other parts of the thesis.
It should not refer to pages of the study, figures, tables, or equations.

The abstract is written in past tense, and passive voice (x was studied) and third person (this study examined) should be preferred over personal pronouns (I, we): First, the text introduces the aim of the study (The objective of the thesis was to...).
Second, it explains the research method in general terms (Qualitative methods were used to...).
Finally, the text ends with the key results and conclusions (The study showed that...).
This part can also discuss whether the thesis succeeded in achieving its set goals. 

The page ends with a field for 3 to 5 keywords accurately describing the thesis content.
Students writing the thesis in Finnish are advised to use the terms of the Finto service.
This service includes some terms in English, but only for few fields of study at \url{http://finto.fi/en/}.
Keywords in English can also be found in field-specific thesauruses, or students can choose the words by themselves after considering the topic thoroughly.
